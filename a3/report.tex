% Created 2020-12-10 Thu 22:47
% Intended LaTeX compiler: pdflatex
\documentclass[11pt]{article}
\usepackage[utf8]{inputenc}
\usepackage[T1]{fontenc}
\usepackage{graphicx}
\usepackage{grffile}
\usepackage{longtable}
\usepackage{wrapfig}
\usepackage{rotating}
\usepackage[normalem]{ulem}
\usepackage{amsmath}
\usepackage{textcomp}
\usepackage{amssymb}
\usepackage{capt-of}
\usepackage{hyperref}
\author{Christian Påbøl}
\date{\today}
\title{Weekly 3}
\hypersetup{
 pdfauthor={Christian Påbøl},
 pdftitle={Weekly 3},
 pdfkeywords={},
 pdfsubject={},
 pdfcreator={Emacs 27.1 (Org mode 9.3)}, 
 pdflang={English}}
\begin{document}

\maketitle
\tableofcontents


\section{ISPC}
\label{sec:org4cc76c5}
For the first part of the weekly assignment, we program with the
Intel Implicit SPMD Program Compiler, and use the \href{https://ispc.github.io/ispc.html}{user guide} for reference. 
To run the assignments and get a runtime comparison, i created a make target
which can be run by using \texttt{make assignment} from the \texttt{src/ispc/} directory.
\subsection{Scan}
\label{sec:org702a113}
First, we try to implement a parallel scan:
\begin{verbatim}
export void scan_ispc(uniform float output[], 
		      uniform float input[],
		      uniform int n) {
    foreach(i = 0 ... n){
    // Exclusive scan
	float in_ = input[i];
	if (programIndex == 0 && i > 0)
	    in_ += output[i-1];

	float tmp = exclusive_scan_add(in_);
	output[i] = shift(tmp, 1);
	if (programIndex == programCount - 1){
	    output[i] = tmp + input[i];
	}
    }
}
\end{verbatim}
This is simple enough, since an inclusive scan is just a shifted exclusive scan.
This is also where we encounter the first differences between ispc and c. When
calculating the scan, we have to account for \(n > programCount\) and carry the
previous output along with us.  
This will be the only task where the C compiler outdoes us, since it automagically
vectorizes the sequential for loop and generates parallel machine code.

\subsection{Pack}
\label{sec:org3628ddb}
Pack was similarly not too tricky:

\begin{verbatim}
export uniform int pack_ispc(uniform int output[],
			     uniform int input[],
			     uniform int n) {
  uniform int m = 0;
  output[0] = input[0];
  foreach (i = 0 ... n) {
    float j = input[i];
    int keep = j != input[i-1];
    int offset = exclusive_scan_add(keep);
    if(!keep){
	offset = programCount -1;
    }
    output[m + offset] = j;
    m += reduce_add(keep);
  }
  return m;
}

\end{verbatim}
This code is heavily based upon the included \texttt{filter.ispc}, except this
time, the \texttt{keep} bool records whether we have found a new unique integer.  
This time, we beat the sequential C compiler, and the code runs 1.5-2 times
faster than its sequential counterpart.

\subsection{Run-length encoding}
\label{sec:orgbbe8b2b}
This is the most tricky problem of the ISPC problems.

\begin{verbatim}
 1  export uniform int rle_ispc(uniform int output[],
 2  			    uniform int input[], 
 3  			    uniform int n) {
 4      // While to handle elements
 5      uniform int c = 0; // Counting offset
 6      uniform int m;
 7      uniform int cur;
 8      uniform int encoded = 0; // use this to handle outputting
 9      while(c < n){
10  	cur = input[c]; // Current element
11  	m = c + 1;
12  	// While to handle counting
13  	while(m <= (n-programCount)){
14  	    // Increment count here
15  	    int element = input[m+programIndex];
16  	    if(all(element == cur)){ // If all elements match
17  		m += programCount; 
18  	    } else {
19  		break;
20  	    }
21  	}
22  	// Once we break we need to check the remaining in the sequence
23  	// A bit hackish count but who cares
24  	while(m < n && input[m] == cur){
25  	    m++;
26  	}
27  	// Handle writing
28  	output[encoded*2] = m - c;
29  	output[encoded*2 + 1] = cur;
30  	encoded += 1;
31  	c = m;
32      }
33      return encoded*2;
34  }
\end{verbatim}

To explain this i go through it section by section
\begin{verbatim}
1  export uniform int rle_ispc(uniform int output[],
2  			    uniform int input[], 
3  			    uniform int n) {
4      // While to handle elements
5      uniform int c = 0; // Counting offset
6      uniform int m;
7      uniform int cur;
8      uniform int encoded = 0; // use this to handle outputtig
\end{verbatim}
This just sets up the values we need to compute

\begin{verbatim}
 9  while(c < n){
10      cur = input[c]; // Current element
11      m = c + 1;
\end{verbatim}
The outer while loop represents the sequence, one iteration per. We start off
a sequence by loading the element in, and choosing where to start. Here, we start
off by immediately counting the value in \texttt{input[c]}. \texttt{m} will be used to keep
track of where we are in the sequence and the final count.

\begin{verbatim}
12  // While to handle counting
13  while(m <= (n-programCount)){
14      // Increment count here
15      int element = input[m+programIndex];
16      if(all(element == cur)){ // If all elements match
17  	m += programCount; 
18      } else {
19  	break;
20      }
21  }
\end{verbatim}
This snippet loads a vector into memory. If all elements in that vector are the same
as the current, then we can just skip ahead to the next vector. This is where the parallelism
comes into play. \texttt{all(element = cur)} Boils down to a vector compare, and allows us to check
8\footnote{on my machine a vector is 8 wide} thus gaining up to 8x speedup. Unfortunately we can't
expect all our sequences to be multiples of ProgramCount, therefore we want to check the 
tail-end of our sequence sequentially.
\begin{verbatim}
22  // Once we break we need to check the remaining in the sequence
23  // A bit hackish count but who cares
24  while(m < n && input[m] == cur){
25      m++;
26  }
\end{verbatim}
Quite simply, this loops m till the sequence breaks, and is guaranteed to run less than
\texttt{programCount} iterations. 
\begin{verbatim}
27  	// Handle writing
28  	output[encoded*2] = m - c;
29  	output[encoded*2 + 1] = cur;
30  	encoded += 1;
31  	c = m;
32      }
33      return encoded*2;
34  }
\end{verbatim}
Finally, knowing where our sequence starts(\texttt{c}) and where it ends(\texttt{m}) we write it out
to memory, and increase our counts, setting the next sequence start to be the end of our current
sequence.

\section{The Halide part}
\label{sec:orga4dd610}
Unfortunately, i didn't have time to complete the halide part of this assignment.
\end{document}
